\problemname{Exponentially Fun Problem}

In your spare time, you've been at \href{https://eta.chalmers.se/}{ETA} building a machine with a
function $S(X) = N$ that given a number $X$ outputs a number $N$, where $N$ is the sum of the prime
factors of $X$ counted with multiplicity. For example, $S(12) = S(2 \cdot 2 \cdot 3) = 2 + 2 + 3 =
7$. Your friend Mateusz wants to try your new machine. He gave the machine a number $X$ but you now
only see the number $N$ on the display of the machine. What is the \textit{smallest} and
\textit{biggest} number $X$ that Mateusz could have given your machine, modulo $10^9 + 7$?

\section*{Input}
Input consists of only one integer $2 \le N \le 30000$.

\section*{Output}
Output two space-separated integers, the smallest possible $X$ that Mateusz could have given your
machine, and the biggest possible $X$, modulo $10^9 + 7$.
